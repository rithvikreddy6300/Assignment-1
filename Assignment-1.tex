\documentclass[a4paper]{article}
\usepackage[utf8]{inputenc}

\title{Assignment-1}
\author{S. RITHVIK REDDY- cs20btech11049}
\date{}
\usepackage{amsmath}
\usepackage{amssymb}
\usepackage{amsfonts}
\usepackage{nopageno}
\usepackage[margin=1in]{geometry}
\usepackage{graphicx}
\usepackage{float}
\usepackage{multicol}
\usepackage{hyperref}
\setlength{\columnsep}{0.5cm}
\setlength{\parindent}{0em}
\usepackage{color}

\usepackage{comment}

\setlength{\columnseprule}{1pt}
\def\columnseprulecolor{\color{black}}

\begin{document}
\maketitle
\noindent
Download all python codes from here

\begin{multicols*}{2}
\noindent
\fbox{%
    \parbox{0.45\textwidth}{%
        \url{https://github.com/rithvikreddy6300/Assignment-1/blob/main/Assignment-1.py}
    }%
    }
    
\vspace{0.3cm}
and latex-tikz codes from  

\vspace{0.3cm}  
    
\fbox{%
    \parbox{0.45\textwidth}{%
        \url{https://github.com/rithvikreddy6300/Assignment-1/blob/main/Assignment-1.tex}
    }%
    }
   
\vspace{0.5cm}
\textbf{QUESTION-4.10}
\vspace{0.5cm}

Two numbers are selected at random (without replacement) from the first six positive integers. Let X denote the larger of the two numbers obtained. Find E(X)?



\vspace{0.5cm}
\textbf{SOLUTION}
\vspace{0.5cm}

The question can be seen as choosing a number first from 1 to 6 numbers and then choosing one more from the remaining 5 numbers, Let $X_1$ be the $1^{st}$ numbers drawn randomly from 1 to 6 and $X_2$be the $2^{nd}$ number drawn from remaining and $X = \text{max } (X_1,X_2)$

\vspace{0.3cm}
$Pr(X_1=n_1)= \begin{cases}
\dfrac{1}{6},  \text{ if } 1 \leq n_1 \leq 6\\
0,  \text{  otherwise }
\end{cases}$

\vspace{0.3cm}
$Pr(X_2=n_2)= \begin{cases}
\dfrac{1}{5},  \text{ if } 1 \leq n_2 \leq 6 \text{ and }n_2 \neq n_1\\
0,  \text{  otherwise }
\end{cases}$

let max $(X_1,X_2)=i$ and Pr(i) denotes the probability that $X = \text{max } (X_1,X_2)=i$
 \begin{multline}
 \noindent
Pr(\text{max } (X_1,X_2)=i)=Pr(X_1=i\text{ and }X_2<i)\\
 +Pr(X_2=i\text{ and }X_1<i) \label{eqn_(0.0.1)}
\end{multline}
 

since choosing of $X_1,X_2$ are independent events, so we can write 
$$Pr(X_1 \text{ and }X_2)=Pr(X_1)Pr(X_2)$$
Substituting this in \eqref{eqn_(0.0.1)} gives us
\begin{multline}
Pr(i)=Pr(X_1=i)Pr(X_2<i)+\\
Pr(X_2=i)Pr(X_1<i)
\end{multline}

$$\Longrightarrow Pr(X=i)=\dfrac{1}{6}\times \dfrac{(i-1)}{5}+\dfrac{(i-1)}{6} \times\dfrac{1}{5}$$
$$\Longrightarrow Pr(X=n)=\dfrac{(i-1)}{15}$$
The expectation value of X represented by E(X) is given by
$$E(X)=\sum_{i=1}^{6} Pr(X=i)\times i$$
\begin{align}
& \Longrightarrow E(X)=\sum_{i=1}^{6} \dfrac{(i-1)}{15}\times i\\
& \Longrightarrow E(X)=\sum_{i=1}^{6} \dfrac{(i^2-i)}{15}\\
& \Longrightarrow E(X)=\dfrac{1}{15} \sum_{i=1}^{6} i^2-\dfrac{1}{15}\sum_{i=1}^{6} i\\
& \Longrightarrow E(X)=\dfrac{1}{15} \times 91-\dfrac{1}{15} \times 21\\
& \Longrightarrow E(X)= \textbf{4.6667}
\end{align}
Therefore the expectation value of X, 

E(X)= \textbf{4.6667}.
\end{multicols*}
\end{document}
