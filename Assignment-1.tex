\documentclass[journal,12pt,twocolumn]{IEEEtran}

\usepackage{setspace}
\usepackage{gensymb}
\singlespacing
\usepackage[cmex10]{amsmath}

\usepackage{amsthm}

\usepackage{mathrsfs}
\usepackage{txfonts}
\usepackage{stfloats}
\usepackage{bm}
\usepackage{cite}
\usepackage{cases}
\usepackage{subfig}

\usepackage{longtable}
\usepackage{multirow}

\usepackage{enumitem}
\usepackage{mathtools}
\usepackage{steinmetz}
\usepackage{tikz}
\usepackage{circuitikz}
\usepackage{verbatim}
\usepackage{tfrupee}
\usepackage[breaklinks=true]{hyperref}
\usepackage{graphicx}
\usepackage{tkz-euclide}

\usetikzlibrary{calc,math}
\usepackage{listings}
    \usepackage{color}                                            %%
    \usepackage{array}                                            %%
    \usepackage{longtable}                                        %%
    \usepackage{calc}                                             %%
    \usepackage{multirow}                                         %%
    \usepackage{hhline}                                           %%
    \usepackage{ifthen}                                           %%
    \usepackage{lscape}     
\usepackage{multicol}
\usepackage{chngcntr}

\DeclareMathOperator*{\Res}{Res}

\renewcommand\thesection{\arabic{section}}
\renewcommand\thesubsection{\thesection.\arabic{subsection}}
\renewcommand\thesubsubsection{\thesubsection.\arabic{subsubsection}}

\renewcommand\thesectiondis{\arabic{section}}
\renewcommand\thesubsectiondis{\thesectiondis.\arabic{subsection}}
\renewcommand\thesubsubsectiondis{\thesubsectiondis.\arabic{subsubsection}}


\hyphenation{op-tical net-works semi-conduc-tor}
\def\inputGnumericTable{}                                 %%

\lstset{
%language=C,
frame=single, 
breaklines=true,
columns=fullflexible
}
\begin{document}


\newtheorem{theorem}{Theorem}[section]
\newtheorem{problem}{Problem}
\newtheorem{proposition}{Proposition}[section]
\newtheorem{lemma}{Lemma}[section]
\newtheorem{corollary}[theorem]{Corollary}
\newtheorem{example}{Example}[section]
\newtheorem{definition}[problem]{Definition}

\newcommand{\BEQA}{\begin{eqnarray}}
\newcommand{\EEQA}{\end{eqnarray}}
\newcommand{\define}{\stackrel{\triangle}{=}}
\bibliographystyle{IEEEtran}
\raggedbottom
\setlength{\parindent}{0pt}
\providecommand{\mbf}{\mathbf}
\providecommand{\pr}[1]{\ensuremath{\Pr\left(#1\right)}}
\providecommand{\qfunc}[1]{\ensuremath{Q\left(#1\right)}}
\providecommand{\sbrak}[1]{\ensuremath{{}\left[#1\right]}}
\providecommand{\lsbrak}[1]{\ensuremath{{}\left[#1\right.}}
\providecommand{\rsbrak}[1]{\ensuremath{{}\left.#1\right]}}
\providecommand{\brak}[1]{\ensuremath{\left(#1\right)}}
\providecommand{\lbrak}[1]{\ensuremath{\left(#1\right.}}
\providecommand{\rbrak}[1]{\ensuremath{\left.#1\right)}}
\providecommand{\cbrak}[1]{\ensuremath{\left\{#1\right\}}}
\providecommand{\lcbrak}[1]{\ensuremath{\left\{#1\right.}}
\providecommand{\rcbrak}[1]{\ensuremath{\left.#1\right\}}}
\theoremstyle{remark}
\newtheorem{rem}{Remark}
\newcommand{\sgn}{\mathop{\mathrm{sgn}}}
\providecommand{\abs}[1]{$\left\vert#1\right\vert$}
\providecommand{\res}[1]{\Res\displaylimits_{#1}} 
\providecommand{\norm}[1]{$\left\lVert#1\right\rVert$}
%\providecommand{\norm}[1]{\lVert#1\rVert}
\providecommand{\mtx}[1]{\mathbf{#1}}
\providecommand{\mean}[1]{E$\left[ #1 \right]$}
\providecommand{\fourier}{\overset{\mathcal{F}}{ \rightleftharpoons}}
%\providecommand{\hilbert}{\overset{\mathcal{H}}{ \rightleftharpoons}}
\providecommand{\system}{\overset{\mathcal{H}}{ \longleftrightarrow}}
	%\newcommand{\solution}[2]{\textbf{Solution:}{#1}}
\newcommand{\solution}{\noindent \textbf{Solution: }}
\newcommand{\cosec}{\,\text{cosec}\,}
\providecommand{\dec}[2]{\ensuremath{\overset{#1}{\underset{#2}{\gtrless}}}}
\newcommand{\myvec}[1]{\ensuremath{\begin{pmatrix}#1\end{pmatrix}}}
\newcommand{\mydet}[1]{\ensuremath{\begin{vmatrix}#1\end{vmatrix}}}
\numberwithin{equation}{subsection}
\makeatletter
\@addtoreset{figure}{problem}
\makeatother
\let\StandardTheFigure\thefigure
\let\vec\mathbf
\renewcommand{\thefigure}{\theproblem}
\def\putbox#1#2#3{\makebox[0in][l]{\makebox[#1][l]{}\raisebox{\baselineskip}[0in][0in]{\raisebox{#2}[0in][0in]{#3}}}}
     \def\rightbox#1{\makebox[0in][r]{#1}}
     \def\centbox#1{\makebox[0in]{#1}}
     \def\topbox#1{\raisebox{-\baselineskip}[0in][0in]{#1}}
     \def\midbox#1{\raisebox{-0.5\baselineskip}[0in][0in]{#1}}
\vspace{3cm}
\title{Assignment 1}
\author{S. Rithvik Reddy - cs20btech11049}
\maketitle
\newpage
\bigskip
\renewcommand{\thefigure}{\theenumi}
\renewcommand{\thetable}{\theenumi}
Download all python codes from 
\begin{lstlisting}
yet to do
\end{lstlisting}
%
and latex-tikz codes from 
%
\begin{lstlisting}
yet to do
\end{lstlisting}
\vspace{0.5cm}
\textbf{QUESTION}
\vspace{0.5cm}

Two numbers are selected at random (without replacement) from the first six positive integers. Let X denote the larger of the two numbers obtained. Find E(X)?

\vspace{0.5cm}
\textbf{SOLUTION}
\vspace{0.5cm}

Let $X_1,X_2$ be the $1^{st},2^{nd}$ numbers drawn randomly from 1 to 6 and X = max $(X_1,X_2)$

let max $(X_1,X_2)$=n let $X_i\in \{ 1,2,3,4,5,6 \}, i=1,2$ so $X \in \{ 1,2,3,4,5,6 \}$, The probability mass function is 

\vspace{0.5cm}
$p_{X_i}(n)= Pr(X_i=n)= \begin{cases}
\dfrac{1}{6},  \test{  if } 1 \leq n \leq 6\\
0,  \test{  otherwise }
\end{cases}$

 $p_X(n) =Pr(max (X_1,X_2)=n)$
 
\begin{equation} \label{eu_eqn}
    = Pr(X_1=n\text{ and }X_2<n)+Pr(X_2=n\text{ and }X_1<n)+Pr(X_1=X_2=n)
\end{equation}
 Since choosing of $X_1,X_2$ are independent events we can write 
 $$Pr(X_1 \text{ and }X_2)=Pr(X_1).Pr(X_2)$$
 Substituting this in (0.0.1) gives us
 $$p_X(n)=Pr(X_1=n).Pr(X_2<n)+Pr(X_2=n).Pr(X_1<n)+Pr(X_1=n).Pr(X_2=n)$$
$$\Longrightarrow p_X(n)=Pr(X=n)=\dfrac{1}{6}.\dfrac{(n-1)}{6}+\dfrac{1}{6}.\dfrac{(n-1)}{6}+\dfrac{1}{6}.\dfrac{1}{6}$$
$$\Longrightarrow p_X(n)=Pr(X=n)=\dfrac{(2n-1)}{36}$$
The expectation value of X represented by E(X) is given by
$$E(x)=\sum_{X=1}^{6} Pr(X=n).X$$
$$\Longrightarrow E(X)=\sum_{X=1}^{6} \dfrac{(2X-1)}{36}.X$$
$$\Longrightarrow E(X)=\sum_{X=1}^{6} \dfrac{(2X^2-X)}{36}$$
$$\Longrightarrow E(X)=\dfrac{2}{36}.\sum_{X=1}^{6} X^2-\dfrac{1}{36}\sum_{X=1}^{6} X$$
$$\Longrightarrow E(X)=\dfrac{2}{36}.91-\dfrac{1}{36}.21$$
$$\Longrightarrow E(X)= 4.4722$$

\end{document}