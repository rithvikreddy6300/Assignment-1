\documentclass[a4paper]{article}
\usepackage[utf8]{inputenc}

\title{Assignment-1}
\author{S. RITHVIK REDDY- cs20btech11049}
\date{}
\usepackage{amsmath}
\usepackage{amssymb}
\usepackage{amsfonts}
\usepackage{nopageno}
\usepackage[margin=0.5in]{geometry}
\usepackage{graphicx}
\usepackage{float}
\usepackage{multicol}
\usepackage{hyperref}
\setlength{\columnsep}{0.5cm}
\setlength{\parindent}{0em}
\usepackage{color}

\usepackage{comment}

\setlength{\columnseprule}{1pt}
\def\columnseprulecolor{\color{black}}

\begin{document}
\maketitle
\noindent
Download all python codes from here

\begin{multicols*}{2}
\noindent
\fbox{%
    \parbox{0.45\textwidth}{%
        \url{https://github.com/rithvikreddy6300/Assignment-1/blob/main/Assignment-1.py}
    }%
    }
    
\vspace{0.3cm}
and latex-tikz codes from  

\vspace{0.3cm}  
    
\fbox{%
    \parbox{0.45\textwidth}{%
        \url{https://github.com/rithvikreddy6300/Assignment-1/blob/main/Assignment-1.tex}
    }%
    }
   
\vspace{0.5cm}
\textbf{QUESTION-4.10}
\vspace{0.5cm}

Two numbers are selected at random (without replacement) from the first six positive integers. Let X denote the larger of the two numbers obtained. Find E(X)?



\vspace{0.5cm}
\textbf{SOLUTION}
\vspace{0.5cm}

Let $X_1,X_2$ be the $1^{st},2^{nd}$ numbers drawn randomly from 1 to 6 and X = max $(X_1,X_2)$

let max $(X_1,X_2)$=n , $X_i\in \{ 1,2,3,4,5,6 \}, i=1,2$ so $X \in \{ 1,2,3,4,5,6 \}$, The probability mass function is 

\vspace{0.5cm}
$p_{X_i}(n)= Pr(X_i=n)= \begin{cases}
\dfrac{1}{6},  \text{ if } 1 \leq n \leq 6 \\
0,  \text{  otherwise }
\end{cases}$

 $p_X(n) =Pr(\text{max } (X_1,X_2)=n)$
 \begin{multline}
 \noindent
 =Pr(X_1=n\text{ and }X_2<n)
 +Pr(X_2=n\text{ and }X_1<n)\\+Pr(X_1=X_2=n)\label{eqn_(0.0.2)}
\end{multline}
 

 Since choosing of $X_1,X_2$ are independent events we can write 
 $$Pr(X_1 \text{ and }X_2)=Pr(X_1).Pr(X_2)$$
 Substituting this in \eqref{eqn_(0.0.2)} gives us
\begin{multline}
p_X(n)=Pr(X_1=n).Pr(X_2<n)+Pr(X_2=n)\\
.Pr(X_1<n)+Pr(X_1=n).Pr(X_2=n)\\
\end{multline}
$$\Longrightarrow p_X(n)=Pr(X=n)=\dfrac{1}{6}.\dfrac{(n-1)}{6}+\dfrac{1}{6}.\dfrac{(n-1)}{6}
+\dfrac{1}{6}.\dfrac{1}{6}$$
$$\Longrightarrow p_X(n)=Pr(X=n)=\dfrac{(2n-1)}{36}$$
The expectation value of X represented by E(X) is given by
$$E(x)=\sum_{X=1}^{6} Pr(X=n).X$$
\begin{align}
& \Longrightarrow E(X)=\sum_{X=1}^{6} \dfrac{(2X-1)}{36}.X\\
& \Longrightarrow E(X)=\sum_{X=1}^{6} \dfrac{(2X^2-X)}{36}\\
& \Longrightarrow E(X)=\dfrac{2}{36}.\sum_{X=1}^{6} X^2-\dfrac{1}{36}\sum_{X=1}^{6} X\\
& \Longrightarrow E(X)=\dfrac{2}{36}.91-\dfrac{1}{36}.21\\
& \Longrightarrow E(X)= \textbf{4.4722}
\end{align}
Therefore the expectation value of X, E(X)= \textbf{4.4722}.
\end{multicols*}
\end{document}